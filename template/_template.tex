\documentclass{beamer}
\usetheme[runningtoc]{mabeamer}
\begin{filecontents*}{bibliography.bib}
@Book{AdamsFournier2003,
	Title                    = {Sobolev spaces},
	Author                   = {Adams, Robert A and Fournier, John JF},
	Year                     = {2003},
}
\end{filecontents*}
\addbibresource{bibliography}

\title{The mabeamer theme}
\date{\today}
\author{Sören Wolfers}

\begin{document}
\matitle*

\section{Main Goals}
\begin{frame}{Light slides}
Light slides do not distract from important content.
\begin{columns}
	\begin{column}{0.5\textwidth}	
		\begin{block}[Done]
		\begin{itemize}
		\item Reduce navigational or structural baggage
		\item Get rid of ugly itemizations	
		\begin{itemize}
		\item First level: small squares
		\item Second level: smaller squares
		\end{itemize}
		\item Provide pretty blocks: plain, equations, alerts, ...
		\end{itemize}
		\end{block}
	\end{column}
	\begin{column}{0.5\textwidth}
		\begin{block}[To Do]
		\begin{itemize}
			\item Allow line breaks in \texttt{equationblock}s; obviate need for \texttt{equationsblock}
		\end{itemize}
		\end{block}
	\end{column}
\end{columns}
\end{frame}

\begin{frame}{Professional slides}
\begin{itemize}
	\item Fancy font lets you forget that these are Beamer slides

	\item Not {\color{purple}using} {\color{orange}playful} {\color{green}colors} {\color{red} protects} {\color{yellow}your} {\color{blue}eyes}
	
\end{itemize}
\end{frame}

\section{Example usage}
\subsection{Standard}

\begin{frame}{Mathematics}
\begin{theorem}
	There are irrational numbers $r$ and $s$ such that $r^s$ is rational.
\end{theorem}
\begin{proof}
	In my opinion, math such as $\sqrt{2}^{\sqrt{2}}$ or 
	$$
	{\sqrt{2}^{\sqrt{2}}}^{\sqrt{2}}=\sqrt{2}^{2}=2
	$$ looks better in serif fonts, even in presentations
\end{proof}
\end{frame}
\begin{frame}{Mathematics}
\begin{itemize}
	\item Important equations can be put in an \texttt{equationblock}:
	\begin{equationblock}
        5^2\approx \int_{0}^{\infty} x^4\exp(-x)\;\mathrm{d}x
	\end{equationblock}	
	\item Equations can be named and are automatically fitted:
	\begin{equationblock}[Rademacher's formula]
		p(n)=\frac{1}{\pi\sqrt{2}}\sum_{k=1}^{\infty}\sqrt{k}A_{k}(n)\frac{\mathrm{d}}{\mathrm{d}n}\left(\frac{1}{\sqrt{n-\frac{1}{24}}}\sinh\left[\frac{\pi}{k}\sqrt{\frac{2}{3}\left(n-\frac{1}{24}\right)}\right]\right)
	\end{equationblock}
	\item Alternatively, \texttt{equationsblock} allows linebreaks but doesn't adjust size:
	\begin{equationsblock}
		\Gamma(n+1)&=n\Gamma(n)\\
		  &=n(n-1)\Gamma(n-1)\\
		  &\dots\\
		  &=n!
	\end{equationsblock}
\end{itemize}
\end{frame}

\begin{frame}[fragile]{Alerts and Listings}
\begin{itemize}
	\item Alerts:
	\begin{alertblock}[\bigexclaim Alarm, Alarm \bigexclaim]
		To alert, use \texttt{alertblocks}. For code snippets use \texttt{lstlisting}. When used in itemizations, make sure you don't indent the code snippet, and use \verb|\begin{frame}[fragile]|
	\end{alertblock}
\item Listings:
	\begin{lstlisting}
\begin{alertblock}[\bigexclaim Alarm, Alarm \bigexclaim]
	To alert, use \texttt{alertblocks}. For code snippets use \texttt{lstlisting}. When used in itemizations, make sure you don't indent the code snippet, and use \verb|\begin{frame}[fragile]|
\end{alertblock}
	\end{lstlisting}
\end{itemize}
\end{frame}

\begin{frame}{Citations and Footnotes}
\begin{itemize}
	\item Foot citations \footcite{AdamsFournier2003} are more slick than in-text citations \citep{AdamsFournier2003}.

	\item Footnotes in slides are questionable but can be created using the footnote command \footnote{Check this out} and use the same counter as foot citations.
\end{itemize}
\end{frame}

\begin{frame}{Enumerations and itemizations}
\begin{enumerate}
	\item One
	\begin{enumerate}
		\item Oneone
		\begin{enumerate}
			\item Oneoneone
		\end{enumerate}
	\end{enumerate}
\end{enumerate}
\begin{itemize}
	\item Square
	\begin{itemize}
		\item Squaresquaresquare
		\begin{itemize}
			\item Squaresquaresquare
		\end{itemize}
	\end{itemize}
\end{itemize}
\begin{itemize}
	\item[\summarize] Nice
	\begin{itemize}
	\item[\summarize] and clean
	\begin{itemize}
		\item[\summarize] and simple
	\end{itemize}
	\end{itemize}
\end{itemize}
\end{frame}
	
\begin{frame}{Figures}
\begin{figure}
	\scalebox{0.6}{\input{runge}}
	\caption{Runge's function and polynomial approximations. Like theorems, figures don't need numbering.}
\end{figure}
\end{frame}

\begin{frame}{Theme options (and tables)}
\begin{table}[]
	\begin{tabularx}{\linewidth}{l>{\raggedright}X}
			\toprule
			\textbf{Option}			& \textbf{Effect} \tabularnewline
			\midrule
			\texttt{sectionhead}		&  Show current section name in frame headers.  \tabularnewline
			\texttt{subsectionhead}  & Show current subsection name in frame headers. \tabularnewline
			\texttt{runningtoc}		& 	Include a table of contents at the beginning of each section and subsection (as on the next slide). \tabularnewline
			\texttt{nocomments} & 	Don't show PDF comments \pdfcomment{<Your comment>} included with \texttt{\textbackslash pdfcomment\{<Your comment>\}}. \tabularnewline
			\bottomrule
	\end{tabularx}
	\caption{Options that can be inserted in \texttt{\textbackslash usetheme[<option>]\{mabeamer\}}.}
\end{table}
\end{frame}

\subsection{Special}

\begin{frame}[fragile]{Additional commands}
\begin{itemize}
	\item For remarks, 	use \texttt{remarkblock}:
    \begin{equationblock}[Pythagoras (right triangles)]
	\highlight{a^2}+\highlight[1]{b^2}=\highlight[2]{c^2}
	\end{equationblock}
	\begin{remarkblock}
		\centering Extension to general triangles is possible but beyond the scope of this presentation
	\end{remarkblock}

	\item For framed blocks, use \texttt{frameblock}:
	\begin{frameblock}
		I'm a simple man. I see Beamer themes, I star.\\
		\phantom{}\hfill ---A simple man
	\end{frameblock}

	\item Use \verb|\matitle*| for title page or \verb|\matitle| for table page plus table of contents
	
	\item Use \verb|equationblock*| for fitted equations without a frame
	\item \emph{Emphasize text} with \verb|\emph|
	\item Highlight math like $\highlight{a^2}$, $\highlight[1]{b^2}$ and $\highlight[2]{c^2}$ with \verb|\highlight<index>|
\end{itemize}
\end{frame}

\end{document}
