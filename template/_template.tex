\documentclass{beamer}
\usetheme[comments,runningtoc]{mabeamer}
\begin{filecontents*}{bibliography.bib}
@Book{AdamsFournier2003,
	Title                    = {Sobolev spaces},
	Author                   = {Adams, Robert A and Fournier, John JF},
	Publisher                = {Academic press},
	Year                     = {2003},
	Volume                   = {140},
}
@Article{AlexanderianPetraStadlerEtAl2016,
	Title                    = {Mean-variance risk-averse optimal control of systems governed by PDEs with random parameter fields using quadratic approximations},
	Author                   = {Alexanderian, Alen and Petra, Noemi and Stadler, Georg and Ghattas, Omar},
	Year 					 = {2016}
}
\end{filecontents*}
\addbibresource{bibliography}

\title{The mabeamer theme}
\date{\today}
\author{Sören Wolfers}

\begin{document}
\matitle*

\section{Main Goals}
\begin{frame}{Light slides}
The main goal of this project is to provide a theme that does not distract from the content.
\begin{columns}
	\begin{column}{0.5\textwidth}	
		\begin{block}{What was done}
		\begin{itemize}
		\item Got rid of navigational or structural baggage
		\item Got rid of ugly itemizations%This comment is necessary to prevent the subitemize to be separated too much. It does so by preventing a newline.
		\begin{itemize}
		\item First level: small squares
		\item Second level: smaller squares
		\end{itemize}
		\item Provided pretty blocks: plain, equations, alerts
		\end{itemize}
		\end{block}
	\end{column}
	\begin{column}{0.5\textwidth}
		\begin{block}{What needs to be done}
		\begin{itemize}
			\item Allow line breaks in \texttt{equationblock}s; obviate need for \texttt{equationsblock}
		\end{itemize}
		\end{block}
	\end{column}
\end{columns}
\end{frame}

\begin{frame}{Professional slides}
\begin{itemize}
	\item Using a non-default font helps forgetting that these are Beamer slides

	\item Not using {\color{yellow}typical} {\color{orange}Beamer} {\color{green}colors} protects your eyes
	
\end{itemize}
\end{frame}

\section{Example usage}
\subsection{Standard}

\begin{frame}{Mathematics}
\begin{theorem}
	There are irrational numbers $r,s$ such that $r^s$ is rational.
\end{theorem}
\begin{proof}
	In my opinion, math such as $\sqrt{2}^{\sqrt{2}}$ or 
	$$
	{\sqrt{2}^{\sqrt{2}}}^{\sqrt{2}}=\sqrt{2}^{2}=2
	$$ looks better in serif fonts, even in presentations
\end{proof}
\end{frame}
\begin{frame}{Mathematics}
\begin{itemize}
	\item Important equations can be put in an \texttt{equationblock}:
	\begin{equationblock}
        5^2\approx \int_{0}^{\infty} x^4\exp(-x)\;\mathrm{d}x
	\end{equationblock}	
	\item Equations can be named and are automatically fitted:
	\begin{equationblock}[Rademacher's formula]
		p(n)=\frac{1}{\pi\sqrt{2}}\sum_{k=1}^{\infty}\sqrt{k}A_{k}(n)\frac{\mathrm{d}}{\mathrm{d}n}\left(\frac{1}{\sqrt{n-\frac{1}{24}}}\sinh\left[\frac{\pi}{k}\sqrt{\frac{2}{3}\left(n-\frac{1}{24}\right)}\right]\right)
	\end{equationblock}
	\item Alternatively, \texttt{equationsblock} allows linebreaks but doesn't adjust size:
	\begin{equationsblock}
		\Gamma(n+1)&=n\Gamma(n)\\
		  &=n(n-1)\Gamma(n-1)\\
		  &\dots\\
		  &=n!
	\end{equationsblock}
\end{itemize}
\end{frame}

\begin{frame}[fragile]{Alerts and Listings}
\begin{itemize}
	\item Alerts:
	\begin{alertblock}[\bigexclaim Alarm, Alarm \bigexclaim]
		To alert, use \texttt{alertblocks}. For code snippets use \texttt{lstlisting}. When used in itemizations, make sure you don't indent the code snippet, and use \verb|\begin{frame}[fragile]|
	\end{alertblock}
\item Listings:
	\begin{lstlisting}
\begin{alertblock}[\bigexclaim Alarm, Alarm \bigexclaim]
	To alert, use \texttt{alertblocks}. For code snippets use \texttt{lstlisting}. When used in itemizations, make sure you don't indent the code snippet, and use \verb|\begin{frame}[fragile]|
\end{alertblock}
	\end{lstlisting}
\end{itemize}
\end{frame}

\begin{frame}{Citations and Footnotes}
\begin{itemize}
	\item Foot citations \footcite{AdamsFournier2003} are better than in-text citations \citep{AlexanderianPetraStadlerEtAl2016}.

	\item Footnotes in slides are questionable but can be created using the footnote command \footnote{Check this out} and get the same counter as foot citations.
\end{itemize}
\end{frame}

\begin{frame}{Enumerations and itemizations}
\begin{enumerate}
	\item Eins
	\begin{enumerate}
		\item Einseins
		\begin{enumerate}
			\item Einseinseins
		\end{enumerate}
	\end{enumerate}
\end{enumerate}
\begin{itemize}
	\item Eins
	\begin{itemize}
		\item Einseins
		\begin{itemize}
			\item Einseinseins
		\end{itemize}
	\end{itemize}
\end{itemize}
\end{frame}
	
\begin{frame}{Figures}
\begin{figure}
	\scalebox{0.6}{\input{runge}}
	\caption{Runge's function and polynomial approximations. Like theorems, figures don't need numbering.}
\end{figure}
\end{frame}

\begin{frame}{Tables}
\begin{table}[]
	\begin{tabularx}{\linewidth}{l>{\raggedright}X}
			\toprule
			\textbf{Option}			& \textbf{Effect} \tabularnewline
			\midrule
			\texttt{sectionhead}		&  Display current section name in frame headers. Analogous option for subsections works too. \tabularnewline
			\texttt{runningtoc}		& 	Include a table of content at the beginning of each section and subsection (as in this template) \tabularnewline
			\texttt{comments} & 	Display PDF comments \pdfcomment{<Your comment>} included with \texttt{\textbackslash pdfcomment\{<Your comment>\}}\tabularnewline
			\bottomrule
	\end{tabularx}
	\caption{Possible options to be inserted in \texttt{\textbackslash usetheme[<option>]\{mabeamer\}}.}
\end{table}
\end{frame}

\subsection{Special}

\begin{frame}[fragile]{Additional commands}
\begin{itemize}
	\item For remarks, 	use \texttt{remarkblock}s:
        \begin{equationblock}[Pythagoras (right triangles)]
		a^2+b^2=c^2
	\end{equationblock}
	\begin{remarkblock}
		\centering Extension to general triangles are possible, but are beyond the scope of this presentation
	\end{remarkblock}

	\item For text blocks, use \texttt{textblock}s:
	\begin{textblock}
		I'm a simple man. I see Beamer themes, I star.\\
		\phantom{}\hfill ---A simple man
	\end{textblock}

	\item Use \verb|\matitle*| for title page or \verb|\matitle| for table page plus table of contents
	
	\item Use \verb|equationblock*| for fitted equations without a frame
\end{itemize}
\end{frame}

\end{document}
