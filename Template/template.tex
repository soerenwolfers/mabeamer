\documentclass[12pt]{beamer}
\usetheme{mabeamer}
\begin{filecontents*}{library.bib}
@Book{AdamsFournier2003,
	Title                    = {Sobolev spaces},
	Author                   = {Adams, Robert A and Fournier, John JF},
	Publisher                = {Academic press},
	Year                     = {2003},
	Volume                   = {140},
}
@Article{AlexanderianPetraStadlerEtAl2016,
	Title                    = {Mean-variance risk-averse optimal control of systems governed by PDEs with random parameter fields using quadratic approximations},
	Author                   = {Alexanderian, Alen and Petra, Noemi and Stadler, Georg and Ghattas, Omar}
}
\end{filecontents*}
\addbibresource{library}

\title{On the mabeamer theme}
\date{\today}%Beamer themes workshop\\[1em] \footnotesize \today
\author[S. Wolfers]{Sören Wolfers}
%\institute[KAUST]{\includegraphics[scale=.35]{KAUST}\vspace{1cm}}
\begin{document}
\frame{\maketitle}
\matableofcontents

\section{Main Goals}
\nosubsections

\begin{frame}{Making it lighter}
The main goal of this project is to provide a theme that does not distract from the content.
\begin{columns}
	\begin{column}{0.5\textwidth}	
		\begin{block}{What  was done}
		\begin{itemize}
		\item Got rid of navigational or structural baggage
		\item Got rid of ugly itemizations
		\begin{itemize}
		\item First level: small squares
		\item Second level: smaller squares
		\end{itemize}
		\end{itemize}
		\end{block}
	\end{column}
	\begin{column}{0.5\textwidth}
		\begin{block}{What  needs to be done}
		\begin{itemize}
			\item Uniform syntax for various blocks
			\item Allow line breaks in \texttt{equationblock}s
		\end{itemize}
		\end{block}
	\end{column}
\end{columns}
\end{frame}

\begin{frame}{Making it modern}
\begin{itemize}
	\item Using a modern font helps forgetting that these are Beamer slides

	\item Not using {\color{yellow}standard} {\color{blue}beamer} {\color{red}colors} makes a more professional impression
\end{itemize}

\begin{exampleblock}{Sections without subsections}
Sections without subsections are ignored in table of contents unless followed by \texttt{\textbackslash nosubsections}
\end{exampleblock}
\end{frame}

\section{Example usage}
\subsection{Standard}
\begin{frame}{Mathematics}
\begin{theorem}
There are irrational numbers $r,s$ such that $r^s$ is rational.
\end{theorem}
\textit{In my opinion, math such as $\sqrt{2}^{\sqrt{2}}$ or 
 $$
 {\sqrt{2}^{\sqrt{2}}}^{\sqrt{2}}=\exp(\sqrt{2}\log\sqrt{2}^{\sqrt{2}})=\exp(\log{2})=2
 $$ looks better when typeset in a math font rather than the usual Beamer font}\hfill\qedsymbol

\end{frame}
\begin{frame}{Mathematics}
\begin{itemize}
	\item Important equations can be set in \emph{\texttt{equationblock}} environments:
	\begin{equationblock}
		5^2\approx 4!=\exp\left(\int x^4\exp(-x)\;dx\right)
	\end{equationblock}	
	\item Long equations are automatically fitted within equationblocks, and can be named:
	\begin{equationblock}[Rademacher's formula]
		p(n)=\frac{1}{\pi\sqrt{2}}\sum_{k=1}^{\infty}\sqrt{k}A_{k}(n)\frac{d}{dn}\left(\frac{1}{\sqrt{n-\frac{1}{24}}}\sinh\left[\frac{\pi}{k}\sqrt{\frac{2}{3}\left(n-\frac{1}{24}\right)}\right]\right)
	\end{equationblock}
\end{itemize}
\end{frame}
\begin{frame}[fragile]{Alerts}
\begin{alertblock}[Alerts]
To alert use \texttt{alertblocks}, for code snippets use \texttt{lstlisting}
\begin{lstlisting}
\begin{alertblock}[Alerts]
To alert use \texttt{alertblocks}, for code snippets use \texttt{lstlisting}
...
\end{alertblock}
\end{lstlisting}

\end{alertblock}
\end{frame}
\begin{frame}{Citations}
One test citation \footcite{AdamsFournier2003} is not as good as two \footcite{AlexanderianPetraStadlerEtAl2016}. 
Zero footnotes\footnote{which can be created using the \\footnote command} are better than one.
\end{frame}
\begin{frame}{Footnotes}
	There is no point in counting footnotes across the presentation\footnote{Right?}
\end{frame}
	
\begin{frame}{Figures}
\begin{figure}
\scalebox{0.6}{% This file was created by matlab2tikz.
%
%The latest updates can be retrieved from
%  http://www.mathworks.com/matlabcentral/fileexchange/22022-matlab2tikz-matlab2tikz
%where you can also make suggestions and rate matlab2tikz.
%
\definecolor{mycolor1}{rgb}{0.00000,0.44700,0.74100}%
\definecolor{mycolor2}{rgb}{0.85000,0.32500,0.09800}%
\definecolor{mycolor3}{rgb}{0.92900,0.69400,0.12500}%
\definecolor{mycolor4}{rgb}{0.49400,0.18400,0.55600}%
%
\begin{tikzpicture}

\begin{axis}[%
width=4.521in,
height=2.352in,
at={(0.758in,0.317in)},
scale only axis,
xmin=-1,
xmax=1,
ymin=-1,
ymax=2,
axis background/.style={fill=white}
]
\addplot [color=mycolor1,solid,forget plot]
  table[row sep=crcr]{%
-1	0.0384615384615385\\
-0.989949748743719	0.0392156668574586\\
-0.979899497487437	0.0399918806413889\\
-0.969849246231156	0.0407910377348773\\
-0.959798994974874	0.0416140374474846\\
-0.949748743718593	0.0424618228528907\\
-0.939698492462312	0.0433353833224268\\
-0.92964824120603	0.0442357572277838\\
-0.919597989949749	0.0451640348256096\\
-0.909547738693467	0.0461213613377652\\
-0.899497487437186	0.0471089402421529\\
-0.889447236180904	0.0481280367902813\\
-0.879396984924623	0.0491799817690934\\
-0.869346733668342	0.0502661755260679\\
-0.85929648241206	0.0513880922782258\\
-0.849246231155779	0.0525472847274377\\
-0.839195979899497	0.0537453890063597\\
-0.829145728643216	0.0549841299814225\\
-0.819095477386935	0.056265326941602\\
-0.809045226130653	0.0575908997041997\\
-0.798994974874372	0.0589628751715985\\
-0.78894472361809	0.0603833943759473\\
-0.778894472361809	0.0618547200519817\\
-0.768844221105528	0.0633792447817472\\
-0.758793969849246	0.0649594997588685\\
-0.748743718592965	0.0665981642242351\\
-0.738693467336683	0.0682980756295854\\
-0.728643216080402	0.0700622405904895\\
-0.718592964824121	0.0718938466956898\\
-0.708542713567839	0.0737962752457018\\
-0.698492462311558	0.0757731150000191\\
-0.688442211055276	0.0778281770192561\\
-0.678391959798995	0.0799655106961266\\
-0.668341708542714	0.0821894210773184\\
-0.658291457286432	0.0845044875871164\\
-0.648241206030151	0.0869155842730661\\
-0.638190954773869	0.0894279017040553\\
-0.628140703517588	0.0920469706619312\\
-0.618090452261307	0.0947786877791234\\
-0.608040201005025	0.0976293432866729\\
-0.597989949748744	0.100605651049473\\
-0.587939698492462	0.103714781078292\\
-0.577889447236181	0.106964394721062\\
-0.5678391959799	0.110362682748742\\
-0.557788944723618	0.113918406563376\\
-0.547738693467337	0.117640942767344\\
-0.537688442211055	0.121540331342496\\
-0.527638190954774	0.125627327695051\\
-0.517587939698492	0.129913458825691\\
-0.507537688442211	0.134411083882617\\
-0.49748743718593	0.139133459346651\\
-0.487437185929648	0.144094809079199\\
-0.477386934673367	0.149310399432936\\
-0.467336683417085	0.154796619577369\\
-0.457286432160804	0.160571067121877\\
-0.447236180904523	0.166652639020983\\
-0.437185929648241	0.173061627612247\\
-0.42713567839196	0.179819821456141\\
-0.417085427135678	0.186950610406655\\
-0.407035175879397	0.194479094025321\\
-0.396984924623116	0.202432192039913\\
-0.386934673366834	0.210838755017942\\
-0.376884422110553	0.219729672744221\\
-0.366834170854271	0.229137976924768\\
-0.35678391959799	0.239098933742287\\
-0.346733668341708	0.249650120409012\\
-0.336683417085427	0.260831478139449\\
-0.326633165829146	0.272685331827634\\
-0.316582914572864	0.285256364081656\\
-0.306532663316583	0.298591528056339\\
-0.296482412060301	0.31273987964557\\
-0.28643216080402	0.327752304967474\\
-0.276381909547739	0.343681113637547\\
-0.266331658291457	0.360579462058165\\
-0.256281407035176	0.378500563913368\\
-0.246231155778894	0.397496637423966\\
-0.236180904522613	0.417617531056883\\
-0.226130653266332	0.438908961940017\\
-0.21608040201005	0.461410295248526\\
-0.206030150753769	0.485151789870874\\
-0.195979899497487	0.510151237987272\\
-0.185929648241206	0.536409936878606\\
-0.175879396984925	0.563907954318913\\
-0.165829145728643	0.592598689132972\\
-0.155778894472362	0.622402791311728\\
-0.14572864321608	0.653201596674694\\
-0.135678391959799	0.684830353128351\\
-0.125628140703518	0.717071669141346\\
-0.115577889447236	0.749649793662212\\
-0.105527638190955	0.782226523920515\\
-0.0954773869346733	0.814399703862131\\
-0.085427135678392	0.845705377354461\\
-0.0753768844221105	0.875624640693406\\
-0.0653266331658291	0.90359603888103\\
-0.0552763819095478	0.929033922957819\\
-0.0452261306532663	0.951352520059578\\
-0.035175879396985	0.969994611277127\\
-0.0251256281407035	0.984462785263262\\
-0.0150753768844221	0.994350424345905\\
-0.00502512562814073	0.999369101095241\\
0.00502512562814061	0.999369101095241\\
0.0150753768844221	0.994350424345905\\
0.0251256281407035	0.984462785263262\\
0.035175879396985	0.969994611277127\\
0.0452261306532664	0.951352520059578\\
0.0552763819095476	0.929033922957819\\
0.0653266331658291	0.90359603888103\\
0.0753768844221105	0.875624640693406\\
0.085427135678392	0.845705377354461\\
0.0954773869346734	0.814399703862131\\
0.105527638190955	0.782226523920516\\
0.115577889447236	0.749649793662212\\
0.125628140703518	0.717071669141346\\
0.135678391959799	0.684830353128351\\
0.14572864321608	0.653201596674694\\
0.155778894472362	0.622402791311728\\
0.165829145728643	0.592598689132972\\
0.175879396984925	0.563907954318913\\
0.185929648241206	0.536409936878606\\
0.195979899497488	0.510151237987272\\
0.206030150753769	0.485151789870875\\
0.21608040201005	0.461410295248526\\
0.226130653266332	0.438908961940017\\
0.236180904522613	0.417617531056883\\
0.246231155778895	0.397496637423965\\
0.256281407035176	0.378500563913367\\
0.266331658291457	0.360579462058165\\
0.276381909547739	0.343681113637547\\
0.28643216080402	0.327752304967474\\
0.296482412060302	0.31273987964557\\
0.306532663316583	0.298591528056339\\
0.316582914572864	0.285256364081656\\
0.326633165829146	0.272685331827634\\
0.336683417085427	0.260831478139449\\
0.346733668341709	0.249650120409012\\
0.35678391959799	0.239098933742287\\
0.366834170854271	0.229137976924768\\
0.376884422110553	0.219729672744221\\
0.386934673366834	0.210838755017942\\
0.396984924623116	0.202432192039913\\
0.407035175879397	0.194479094025321\\
0.417085427135678	0.186950610406655\\
0.42713567839196	0.179819821456141\\
0.437185929648241	0.173061627612247\\
0.447236180904523	0.166652639020983\\
0.457286432160804	0.160571067121877\\
0.467336683417085	0.154796619577369\\
0.477386934673367	0.149310399432936\\
0.487437185929648	0.144094809079199\\
0.49748743718593	0.139133459346651\\
0.507537688442211	0.134411083882617\\
0.517587939698492	0.129913458825691\\
0.527638190954774	0.125627327695051\\
0.537688442211055	0.121540331342496\\
0.547738693467337	0.117640942767344\\
0.557788944723618	0.113918406563376\\
0.567839195979899	0.110362682748742\\
0.577889447236181	0.106964394721062\\
0.587939698492462	0.103714781078292\\
0.597989949748744	0.100605651049473\\
0.608040201005025	0.0976293432866729\\
0.618090452261306	0.0947786877791234\\
0.628140703517588	0.0920469706619312\\
0.638190954773869	0.0894279017040553\\
0.648241206030151	0.0869155842730661\\
0.658291457286432	0.0845044875871164\\
0.668341708542713	0.0821894210773184\\
0.678391959798995	0.0799655106961266\\
0.688442211055276	0.0778281770192561\\
0.698492462311558	0.0757731150000191\\
0.708542713567839	0.0737962752457018\\
0.71859296482412	0.0718938466956898\\
0.728643216080402	0.0700622405904895\\
0.738693467336683	0.0682980756295854\\
0.748743718592965	0.0665981642242351\\
0.758793969849246	0.0649594997588685\\
0.768844221105528	0.0633792447817472\\
0.778894472361809	0.0618547200519817\\
0.78894472361809	0.0603833943759473\\
0.798994974874372	0.0589628751715985\\
0.809045226130653	0.0575908997041997\\
0.819095477386935	0.056265326941602\\
0.829145728643216	0.0549841299814225\\
0.839195979899497	0.0537453890063597\\
0.849246231155779	0.0525472847274377\\
0.85929648241206	0.0513880922782258\\
0.869346733668342	0.0502661755260679\\
0.879396984924623	0.0491799817690934\\
0.889447236180904	0.0481280367902813\\
0.899497487437186	0.0471089402421529\\
0.909547738693467	0.0461213613377652\\
0.919597989949749	0.0451640348256096\\
0.92964824120603	0.0442357572277838\\
0.939698492462312	0.0433353833224268\\
0.949748743718593	0.0424618228528907\\
0.959798994974874	0.0416140374474846\\
0.969849246231156	0.0407910377348773\\
0.979899497487437	0.0399918806413889\\
0.989949748743719	0.0392156668574586\\
1	0.0384615384615385\\
};
\addplot [color=mycolor2,solid,forget plot]
  table[row sep=crcr]{%
-1	0.0384615384615385\\
-0.989949748743719	0.156662323392184\\
-0.979899497487437	0.259753787931321\\
-0.969849246231156	0.348761937665585\\
-0.959798994974874	0.424671584904322\\
-0.949748743718593	0.488427294028285\\
-0.939698492462312	0.540934317092457\\
-0.92964824120603	0.583059519683029\\
-0.919597989949749	0.615632297028495\\
-0.909547738693467	0.6394454803649\\
-0.899497487437186	0.655256233555216\\
-0.889447236180904	0.663786939962861\\
-0.879396984924623	0.665726079579347\\
-0.869346733668342	0.661729096406078\\
-0.85929648241206	0.652419256090273\\
-0.849246231155779	0.638388493815032\\
-0.839195979899497	0.620198252443536\\
-0.829145728643216	0.598380310917394\\
-0.819095477386935	0.573437602909107\\
-0.809045226130653	0.545845025728694\\
-0.798994974874372	0.516050239484437\\
-0.78894472361809	0.484474456497768\\
-0.778894472361809	0.451513220972298\\
-0.768844221105528	0.417537178916977\\
-0.758793969849246	0.382892838323392\\
-0.748743718592965	0.347903319597209\\
-0.738693467336683	0.312869096243743\\
-0.728643216080402	0.27806872580767\\
-0.718592964824121	0.243759571066875\\
-0.708542713567839	0.210178511480439\\
-0.698492462311558	0.17754264489076\\
-0.688442211055276	0.146049979479816\\
-0.678391959798995	0.115880115979556\\
-0.668341708542714	0.0871949201364428\\
-0.658291457286432	0.0601391854301159\\
-0.648241206030151	0.0348412860462057\\
-0.638190954773869	0.0114138201032783\\
-0.628140703517588	-0.0100457568660825\\
-0.618090452261307	-0.0294545081800544\\
-0.608040201005025	-0.0467434020182195\\
-0.597989949748744	-0.0618567071780694\\
-0.587939698492462	-0.0747513985773733\\
-0.577889447236181	-0.0853965725024094\\
-0.5678391959799	-0.0937728716020585\\
-0.557788944723618	-0.0998719196277612\\
-0.547738693467337	-0.103695765919337\\
-0.537688442211055	-0.105256339636666\\
-0.527638190954774	-0.104574913737235\\
-0.517587939698492	-0.101681578699545\\
-0.507537688442211	-0.096614725992383\\
-0.49748743718593	-0.0894205412899534\\
-0.487437185929648	-0.0801525074328759\\
-0.477386934673367	-0.0688709171350445\\
-0.467336683417085	-0.05564239543635\\
-0.457286432160804	-0.0405394319012633\\
-0.447236180904523	-0.023639922563285\\
-0.437185929648241	-0.00502672161525417\\
-0.42713567839196	0.0152127971544774\\
-0.417085427135678	0.0369871691800087\\
-0.407035175879397	0.0602012581899924\\
-0.396984924623116	0.0847566655338707\\
-0.386934673366834	0.110552129762253\\
-0.376884422110553	0.137483916461427\\
-0.366834170854271	0.165446198342009\\
-0.35678391959799	0.19433142558173\\
-0.346733668341708	0.224030686422365\\
-0.336683417085427	0.254434058020783\\
-0.326633165829146	0.28543094755416\\
-0.316582914572864	0.316910423579304\\
-0.306532663316583	0.34876153764613\\
-0.296482412060301	0.380873636165276\\
-0.28643216080402	0.413136662529841\\
-0.276381909547739	0.445441449491278\\
-0.266331658291457	0.477680001789409\\
-0.256281407035176	0.509745769036589\\
-0.246231155778894	0.541533908856\\
-0.236180904522613	0.572941540274081\\
-0.226130653266332	0.603867987367106\\
-0.21608040201005	0.634215013161883\\
-0.206030150753769	0.663887043790605\\
-0.195979899497487	0.69279138289983\\
-0.185929648241206	0.720838416313597\\
-0.175879396984925	0.747941806950689\\
-0.165829145728643	0.77401867999602\\
-0.155778894472362	0.798989798326169\\
-0.14572864321608	0.822779728189049\\
-0.135678391959799	0.845316995137707\\
-0.125628140703518	0.866534230218273\\
-0.115577889447236	0.886368306412033\\
-0.105527638190955	0.904760465331651\\
-0.0954773869346733	0.921656434171519\\
-0.085427135678392	0.937006532912249\\
-0.0753768844221105	0.950765771779302\\
-0.0653266331658291	0.962893938955754\\
-0.0552763819095478	0.973355678549195\\
-0.0452261306532663	0.982120558812773\\
-0.035175879396985	0.989163130620368\\
-0.0251256281407035	0.994462976195908\\
-0.0150753768844221	0.998004748096818\\
-0.00502512562814073	0.99977819845161\\
0.00502512562814061	0.99977819845161\\
0.0150753768844221	0.998004748096818\\
0.0251256281407035	0.994462976195908\\
0.035175879396985	0.989163130620368\\
0.0452261306532664	0.982120558812773\\
0.0552763819095476	0.973355678549195\\
0.0653266331658291	0.962893938955754\\
0.0753768844221105	0.950765771779302\\
0.085427135678392	0.937006532912249\\
0.0954773869346734	0.921656434171519\\
0.105527638190955	0.904760465331652\\
0.115577889447236	0.886368306412033\\
0.125628140703518	0.866534230218273\\
0.135678391959799	0.845316995137707\\
0.14572864321608	0.822779728189049\\
0.155778894472362	0.798989798326169\\
0.165829145728643	0.77401867999602\\
0.175879396984925	0.747941806950689\\
0.185929648241206	0.720838416313597\\
0.195979899497488	0.692791382899829\\
0.206030150753769	0.663887043790605\\
0.21608040201005	0.634215013161883\\
0.226130653266332	0.603867987367106\\
0.236180904522613	0.572941540274081\\
0.246231155778895	0.541533908855999\\
0.256281407035176	0.509745769036589\\
0.266331658291457	0.477680001789409\\
0.276381909547739	0.445441449491277\\
0.28643216080402	0.413136662529841\\
0.296482412060302	0.380873636165276\\
0.306532663316583	0.34876153764613\\
0.316582914572864	0.316910423579304\\
0.326633165829146	0.28543094755416\\
0.336683417085427	0.254434058020783\\
0.346733668341709	0.224030686422364\\
0.35678391959799	0.19433142558173\\
0.366834170854271	0.165446198342009\\
0.376884422110553	0.137483916461427\\
0.386934673366834	0.110552129762253\\
0.396984924623116	0.0847566655338707\\
0.407035175879397	0.0602012581899921\\
0.417085427135678	0.036987169180009\\
0.42713567839196	0.0152127971544774\\
0.437185929648241	-0.00502672161525406\\
0.447236180904523	-0.023639922563285\\
0.457286432160804	-0.0405394319012635\\
0.467336683417085	-0.0556423954363498\\
0.477386934673367	-0.0688709171350446\\
0.487437185929648	-0.0801525074328758\\
0.49748743718593	-0.0894205412899534\\
0.507537688442211	-0.096614725992383\\
0.517587939698492	-0.101681578699545\\
0.527638190954774	-0.104574913737235\\
0.537688442211055	-0.105256339636666\\
0.547738693467337	-0.103695765919337\\
0.557788944723618	-0.0998719196277612\\
0.567839195979899	-0.0937728716020586\\
0.577889447236181	-0.0853965725024092\\
0.587939698492462	-0.0747513985773732\\
0.597989949748744	-0.0618567071780693\\
0.608040201005025	-0.0467434020182194\\
0.618090452261306	-0.0294545081800546\\
0.628140703517588	-0.0100457568660826\\
0.638190954773869	0.0114138201032783\\
0.648241206030151	0.0348412860462057\\
0.658291457286432	0.0601391854301159\\
0.668341708542713	0.0871949201364425\\
0.678391959798995	0.115880115979556\\
0.688442211055276	0.146049979479816\\
0.698492462311558	0.17754264489076\\
0.708542713567839	0.210178511480439\\
0.71859296482412	0.243759571066874\\
0.728643216080402	0.278068725807669\\
0.738693467336683	0.312869096243743\\
0.748743718592965	0.347903319597209\\
0.758793969849246	0.382892838323392\\
0.768844221105528	0.417537178916977\\
0.778894472361809	0.451513220972298\\
0.78894472361809	0.484474456497768\\
0.798994974874372	0.516050239484437\\
0.809045226130653	0.545845025728694\\
0.819095477386935	0.573437602909107\\
0.829145728643216	0.598380310917393\\
0.839195979899497	0.620198252443537\\
0.849246231155779	0.638388493815032\\
0.85929648241206	0.652419256090274\\
0.869346733668342	0.661729096406079\\
0.879396984924623	0.665726079579347\\
0.889447236180904	0.663786939962861\\
0.899497487437186	0.655256233555217\\
0.909547738693467	0.6394454803649\\
0.919597989949749	0.615632297028495\\
0.92964824120603	0.583059519683029\\
0.939698492462312	0.540934317092457\\
0.949748743718593	0.488427294028285\\
0.959798994974874	0.424671584904322\\
0.969849246231156	0.348761937665584\\
0.979899497487437	0.259753787931322\\
0.989949748743719	0.156662323392184\\
1	0.0384615384615385\\
};
\addplot [color=mycolor3,solid,forget plot]
  table[row sep=crcr]{%
-1	0.0384615384615385\\
-0.989949748743719	-0.249266628796471\\
-0.979899497487437	-0.48010807143389\\
-0.969849246231156	-0.660545628872702\\
-0.959798994974874	-0.796561091008026\\
-0.949748743718593	-0.893662078720843\\
-0.939698492462312	-0.95690795144556\\
-0.92964824120603	-0.990934763424421\\
-0.919597989949749	-0.999979290055268\\
-0.909547738693467	-0.987902145513931\\
-0.899497487437186	-0.958210012607161\\
-0.889447236180904	-0.914077005586666\\
-0.879396984924623	-0.858365186429513\\
-0.869346733668342	-0.793644254864815\\
-0.85929648241206	-0.722210432201264\\
-0.849246231155779	-0.646104558784784\\
-0.839195979899497	-0.567129424690214\\
-0.829145728643216	-0.486866353025595\\
-0.819095477386935	-0.406691055002334\\
-0.809045226130653	-0.327788775699149\\
-0.798994974874372	-0.251168749222395\\
-0.78894472361809	-0.177677981740017\\
-0.778894472361809	-0.108014380641068\\
-0.768844221105528	-0.0427392478473609\\
-0.758793969849246	0.017710844921462\\
-0.748743718592965	0.0730127813535347\\
-0.738693467336683	0.122946208870441\\
-0.728643216080402	0.167383412947705\\
-0.718592964824121	0.206279740530161\\
-0.708542713567839	0.239664555867612\\
-0.698492462311558	0.267632712321508\\
-0.688442211055276	0.29033652391871\\
-0.678391959798995	0.307978220653753\\
-0.668341708542714	0.320802871766317\\
-0.658291457286432	0.329091761446003\\
-0.648241206030151	0.333156201641776\\
-0.638190954773869	0.333331766878835\\
-0.628140703517588	0.329972936210956\\
-0.618090452261307	0.323448127661703\\
-0.608040201005025	0.314135110733243\\
-0.597989949748744	0.302416782786816\\
-0.587939698492462	0.288677295324253\\
-0.577889447236181	0.273298516425274\\
-0.5678391959799	0.256656815820607\\
-0.557788944723618	0.239120159306338\\
-0.547738693467337	0.22104549943019\\
-0.537688442211055	0.202776449605806\\
-0.527638190954774	0.184641229036408\\
-0.517587939698492	0.166950866054554\\
-0.507537688442211	0.14999764771005\\
-0.49748743718593	0.134053803663392\\
-0.487437185929648	0.119370412667457\\
-0.477386934673367	0.106176520145494\\
-0.467336683417085	0.0946784555987884\\
-0.457286432160804	0.0850593388027215\\
-0.447236180904523	0.0774787639752642\\
-0.437185929648241	0.0720726513272827\\
-0.42713567839196	0.0689532556293703\\
-0.417085427135678	0.0682093216552432\\
-0.407035175879397	0.0699063765870791\\
-0.396984924623116	0.074087149693502\\
-0.386934673366834	0.0807721098162558\\
-0.376884422110553	0.0899601114269375\\
-0.366834170854271	0.101629140240495\\
-0.35678391959799	0.115737149597522\\
-0.346733668341708	0.132222979052732\\
-0.336683417085427	0.151007346832291\\
-0.326633165829146	0.171993908048069\\
-0.316582914572864	0.195070370782151\\
-0.306532663316583	0.22010966238033\\
-0.296482412060301	0.246971138518593\\
-0.28643216080402	0.275501827831976\\
-0.276381909547739	0.305537705120483\\
-0.266331658291457	0.336904986372087\\
-0.256281407035176	0.369421439068186\\
-0.246231155778894	0.402897701462206\\
-0.236180904522613	0.43713860474736\\
-0.226130653266332	0.471944492254951\\
-0.21608040201005	0.507112530049875\\
-0.206030150753769	0.542438003515376\\
-0.195979899497487	0.577715594744384\\
-0.185929648241206	0.612740635780149\\
-0.175879396984925	0.647310332974162\\
-0.165829145728643	0.681224957954741\\
-0.155778894472362	0.714289000924947\\
-0.14572864321608	0.746312282233864\\
-0.135678391959799	0.777111018390571\\
-0.125628140703518	0.806508838915519\\
-0.115577889447236	0.834337750649291\\
-0.105527638190955	0.860439046364116\\
-0.0954773869346733	0.884664154748815\\
-0.085427135678392	0.906875429063172\\
-0.0753768844221105	0.9269468719831\\
-0.0653266331658291	0.94476479438325\\
-0.0552763819095478	0.960228406029099\\
-0.0452261306532663	0.973250336375833\\
-0.035175879396985	0.983757083896723\\
-0.0251256281407035	0.991689392588968\\
-0.0150753768844221	0.997002554530385\\
-0.00502512562814073	0.999666637585569\\
0.00502512562814061	0.999666637585569\\
0.0150753768844221	0.997002554530385\\
0.0251256281407035	0.991689392588968\\
0.035175879396985	0.983757083896723\\
0.0452261306532664	0.973250336375833\\
0.0552763819095476	0.960228406029099\\
0.0653266331658291	0.94476479438325\\
0.0753768844221105	0.9269468719831\\
0.085427135678392	0.906875429063172\\
0.0954773869346734	0.884664154748814\\
0.105527638190955	0.860439046364116\\
0.115577889447236	0.834337750649291\\
0.125628140703518	0.806508838915519\\
0.135678391959799	0.777111018390571\\
0.14572864321608	0.746312282233863\\
0.155778894472362	0.714289000924948\\
0.165829145728643	0.681224957954741\\
0.175879396984925	0.647310332974162\\
0.185929648241206	0.612740635780149\\
0.195979899497488	0.577715594744384\\
0.206030150753769	0.542438003515376\\
0.21608040201005	0.507112530049876\\
0.226130653266332	0.471944492254951\\
0.236180904522613	0.43713860474736\\
0.246231155778895	0.402897701462205\\
0.256281407035176	0.369421439068186\\
0.266331658291457	0.336904986372087\\
0.276381909547739	0.305537705120483\\
0.28643216080402	0.275501827831976\\
0.296482412060302	0.246971138518592\\
0.306532663316583	0.220109662380329\\
0.316582914572864	0.195070370782151\\
0.326633165829146	0.171993908048069\\
0.336683417085427	0.151007346832291\\
0.346733668341709	0.132222979052732\\
0.35678391959799	0.115737149597522\\
0.366834170854271	0.101629140240495\\
0.376884422110553	0.0899601114269376\\
0.386934673366834	0.0807721098162557\\
0.396984924623116	0.0740871496935021\\
0.407035175879397	0.0699063765870791\\
0.417085427135678	0.0682093216552432\\
0.42713567839196	0.0689532556293703\\
0.437185929648241	0.0720726513272826\\
0.447236180904523	0.0774787639752642\\
0.457286432160804	0.0850593388027216\\
0.467336683417085	0.0946784555987882\\
0.477386934673367	0.106176520145494\\
0.487437185929648	0.119370412667457\\
0.49748743718593	0.134053803663392\\
0.507537688442211	0.14999764771005\\
0.517587939698492	0.166950866054554\\
0.527638190954774	0.184641229036408\\
0.537688442211055	0.202776449605806\\
0.547738693467337	0.22104549943019\\
0.557788944723618	0.239120159306338\\
0.567839195979899	0.256656815820607\\
0.577889447236181	0.273298516425274\\
0.587939698492462	0.288677295324253\\
0.597989949748744	0.302416782786816\\
0.608040201005025	0.314135110733243\\
0.618090452261306	0.323448127661703\\
0.628140703517588	0.329972936210956\\
0.638190954773869	0.333331766878835\\
0.648241206030151	0.333156201641776\\
0.658291457286432	0.329091761446003\\
0.668341708542713	0.320802871766317\\
0.678391959798995	0.307978220653753\\
0.688442211055276	0.290336523918711\\
0.698492462311558	0.267632712321508\\
0.708542713567839	0.239664555867612\\
0.71859296482412	0.206279740530161\\
0.728643216080402	0.167383412947705\\
0.738693467336683	0.122946208870441\\
0.748743718592965	0.0730127813535347\\
0.758793969849246	0.017710844921462\\
0.768844221105528	-0.0427392478473616\\
0.778894472361809	-0.108014380641068\\
0.78894472361809	-0.177677981740017\\
0.798994974874372	-0.251168749222395\\
0.809045226130653	-0.32778877569915\\
0.819095477386935	-0.406691055002335\\
0.829145728643216	-0.486866353025595\\
0.839195979899497	-0.567129424690214\\
0.849246231155779	-0.646104558784784\\
0.85929648241206	-0.722210432201264\\
0.869346733668342	-0.793644254864816\\
0.879396984924623	-0.858365186429513\\
0.889447236180904	-0.914077005586665\\
0.899497487437186	-0.958210012607161\\
0.909547738693467	-0.987902145513931\\
0.919597989949749	-0.999979290055268\\
0.92964824120603	-0.990934763424422\\
0.939698492462312	-0.956907951445561\\
0.949748743718593	-0.893662078720843\\
0.959798994974874	-0.796561091008027\\
0.969849246231156	-0.6605456288727\\
0.979899497487437	-0.480108071433893\\
0.989949748743719	-0.249266628796472\\
1	0.0384615384615385\\
};
\addplot [color=mycolor4,solid,forget plot]
  table[row sep=crcr]{%
-1	0.0384615384615385\\
-0.989949748743719	0.729036851708777\\
-0.979899497487437	1.2345580104463\\
-0.969849246231156	1.58523833845631\\
-0.959798994974874	1.80773572379773\\
-0.949748743718593	1.92546757763324\\
-0.939698492462312	1.95890471361276\\
-0.92964824120603	1.92584519164973\\
-0.919597989949749	1.84166913356037\\
-0.909547738693467	1.71957548247109\\
-0.899497487437186	1.57080164312625\\
-0.889447236180904	1.40482690623983\\
-0.879396984924623	1.22956052682123\\
-0.869346733668342	1.05151529395914\\
-0.85929648241206	0.875967397859925\\
-0.849246231155779	0.707103368999761\\
-0.839195979899497	0.548154834054226\\
-0.829145728643216	0.401521803806904\\
-0.819095477386935	0.26888517950149\\
-0.809045226130653	0.151309136081246\\
-0.798994974874372	0.0493340134470667\\
-0.78894472361809	-0.0369396797473715\\
-0.778894472361809	-0.107775571166628\\
-0.768844221105528	-0.163733485372706\\
-0.758793969849246	-0.205608731118992\\
-0.748743718592965	-0.234378110463926\\
-0.738693467336683	-0.251152168050496\\
-0.728643216080402	-0.257133221489487\\
-0.718592964824121	-0.253578735735756\\
-0.708542713567839	-0.2417696256658\\
-0.698492462311558	-0.222983091760175\\
-0.688442211055276	-0.198469613874181\\
-0.678391959798995	-0.169433747553199\\
-0.668341708542714	-0.137018386223499\\
-0.658291457286432	-0.102292170873714\\
-0.648241206030151	-0.0662397465448992\\
-0.638190954773869	-0.0297545820765966\\
-0.628140703517588	0.00636591387895884\\
-0.618090452261307	0.0414232315539808\\
-0.608040201005025	0.0748187889926891\\
-0.597989949748744	0.10605337180741\\
-0.587939698492462	0.134725189971482\\
-0.577889447236181	0.160526743989997\\
-0.5678391959799	0.183240679436397\\
-0.557788944723618	0.202734795995695\\
-0.547738693467337	0.218956364805211\\
-0.537688442211055	0.231925896022733\\
-0.527638190954774	0.241730487171546\\
-0.517587939698492	0.248516871903346\\
-0.507537688442211	0.252484278375274\\
-0.49748743718593	0.253877196447713\\
-0.487437185929648	0.252978143366689\\
-0.477386934673367	0.250100508490234\\
-0.467336683417085	0.245581548943524\\
-0.457286432160804	0.239775599834516\\
-0.447236180904523	0.233047554821791\\
-0.437185929648241	0.225766665390921\\
-0.42713567839196	0.218300700156465\\
-0.417085427135678	0.211010498855242\\
-0.407035175879397	0.204244949424471\\
-0.396984924623116	0.198336410657109\\
-0.386934673366834	0.193596597388036\\
-0.376884422110553	0.190312939980048\\
-0.366834170854271	0.188745425039527\\
-0.35678391959799	0.189123919789825\\
-0.346733668341708	0.191645978357246\\
-0.336683417085427	0.196475124371724\\
-0.326633165829146	0.20373960074341\\
-0.316582914572864	0.213531574238935\\
-0.306532663316583	0.225906779538739\\
-0.296482412060301	0.240884584801077\\
-0.28643216080402	0.258448458380685\\
-0.276381909547739	0.278546814242278\\
-0.266331658291457	0.301094211762469\\
-0.256281407035176	0.325972884020091\\
-0.246231155778894	0.353034567325704\\
-0.236180904522613	0.382102603627903\\
-0.226130653266332	0.412974286548526\\
-0.21608040201005	0.445423421132432\\
-0.206030150753769	0.479203066941911\\
-0.195979899497487	0.514048433872451\\
-0.185929648241206	0.549679900007134\\
-0.175879396984925	0.585806120952951\\
-0.165829145728643	0.622127200405327\\
-0.155778894472362	0.658337892158812\\
-0.14572864321608	0.694130804413644\\
-0.135678391959799	0.729199578011406\\
-0.125628140703518	0.763242011159872\\
-0.115577889447236	0.795963104268755\\
-0.105527638190955	0.827077999706296\\
-0.0954773869346733	0.856314792592722\\
-0.085427135678392	0.883417190162394\\
-0.0753768844221105	0.908146998743329\\
-0.0653266331658291	0.930286419012442\\
-0.0552763819095478	0.949640131878727\\
-0.0452261306532663	0.966037159116446\\
-0.035175879396985	0.979332484707563\\
-0.0251256281407035	0.989408424748937\\
-0.0150753768844221	0.996175735726579\\
-0.00502512562814073	0.99957445294823\\
0.00502512562814061	0.99957445294823\\
0.0150753768844221	0.99617573572658\\
0.0251256281407035	0.989408424748938\\
0.035175879396985	0.979332484707563\\
0.0452261306532664	0.966037159116446\\
0.0552763819095476	0.949640131878728\\
0.0653266331658291	0.930286419012442\\
0.0753768844221105	0.90814699874333\\
0.085427135678392	0.883417190162394\\
0.0954773869346734	0.856314792592722\\
0.105527638190955	0.827077999706297\\
0.115577889447236	0.795963104268755\\
0.125628140703518	0.763242011159872\\
0.135678391959799	0.729199578011406\\
0.14572864321608	0.694130804413644\\
0.155778894472362	0.658337892158813\\
0.165829145728643	0.622127200405327\\
0.175879396984925	0.585806120952951\\
0.185929648241206	0.549679900007134\\
0.195979899497488	0.51404843387245\\
0.206030150753769	0.479203066941911\\
0.21608040201005	0.445423421132433\\
0.226130653266332	0.412974286548526\\
0.236180904522613	0.382102603627902\\
0.246231155778895	0.353034567325703\\
0.256281407035176	0.325972884020091\\
0.266331658291457	0.301094211762469\\
0.276381909547739	0.278546814242278\\
0.28643216080402	0.258448458380685\\
0.296482412060302	0.240884584801077\\
0.306532663316583	0.225906779538739\\
0.316582914572864	0.213531574238935\\
0.326633165829146	0.20373960074341\\
0.336683417085427	0.196475124371724\\
0.346733668341709	0.191645978357246\\
0.35678391959799	0.189123919789825\\
0.366834170854271	0.188745425039527\\
0.376884422110553	0.190312939980048\\
0.386934673366834	0.193596597388036\\
0.396984924623116	0.198336410657109\\
0.407035175879397	0.204244949424471\\
0.417085427135678	0.211010498855242\\
0.42713567839196	0.218300700156465\\
0.437185929648241	0.225766665390921\\
0.447236180904523	0.23304755482179\\
0.457286432160804	0.239775599834516\\
0.467336683417085	0.245581548943524\\
0.477386934673367	0.250100508490234\\
0.487437185929648	0.252978143366689\\
0.49748743718593	0.253877196447713\\
0.507537688442211	0.252484278375274\\
0.517587939698492	0.248516871903346\\
0.527638190954774	0.241730487171546\\
0.537688442211055	0.231925896022733\\
0.547738693467337	0.218956364805211\\
0.557788944723618	0.202734795995695\\
0.567839195979899	0.183240679436397\\
0.577889447236181	0.160526743989997\\
0.587939698492462	0.134725189971482\\
0.597989949748744	0.10605337180741\\
0.608040201005025	0.074818788992689\\
0.618090452261306	0.0414232315539811\\
0.628140703517588	0.00636591387895884\\
0.638190954773869	-0.0297545820765966\\
0.648241206030151	-0.0662397465448989\\
0.658291457286432	-0.102292170873714\\
0.668341708542713	-0.137018386223499\\
0.678391959798995	-0.169433747553199\\
0.688442211055276	-0.198469613874181\\
0.698492462311558	-0.222983091760175\\
0.708542713567839	-0.2417696256658\\
0.71859296482412	-0.253578735735755\\
0.728643216080402	-0.257133221489487\\
0.738693467336683	-0.251152168050496\\
0.748743718592965	-0.234378110463926\\
0.758793969849246	-0.205608731118992\\
0.768844221105528	-0.163733485372706\\
0.778894472361809	-0.107775571166628\\
0.78894472361809	-0.0369396797473714\\
0.798994974874372	0.0493340134470667\\
0.809045226130653	0.151309136081246\\
0.819095477386935	0.268885179501492\\
0.829145728643216	0.401521803806903\\
0.839195979899497	0.548154834054226\\
0.849246231155779	0.707103368999761\\
0.85929648241206	0.875967397859925\\
0.869346733668342	1.05151529395914\\
0.879396984924623	1.22956052682123\\
0.889447236180904	1.40482690623982\\
0.899497487437186	1.57080164312625\\
0.909547738693467	1.7195754824711\\
0.919597989949749	1.84166913356037\\
0.92964824120603	1.92584519164972\\
0.939698492462312	1.95890471361276\\
0.949748743718593	1.92546757763324\\
0.959798994974874	1.80773572379773\\
0.969849246231156	1.5852383384563\\
0.979899497487437	1.2345580104463\\
0.989949748743719	0.729036851708776\\
1	0.0384615384615385\\
};
\end{axis}
\end{tikzpicture}%}
\caption{Runge's function and polynomial approximations. Observe that figures don't need numbering, just as Theorems.}
\end{figure}

\end{frame}
\begin{frame}{Lists}
\begin{enumerate}
\item Eins
\begin{enumerate}
\item Einseins
\begin{enumerate}
\item Einseinseins
\end{enumerate}
\end{enumerate}
\end{enumerate}
\end{frame}
\begin{frame}{Tables}
\begin{table}[]
\begin{tabularx}{\linewidth}{l>{\raggedright}X}
		\toprule
		\textbf{Option}			& \textbf{Auswirkung} \tabularnewline
		\midrule
		\texttt{noflama}		& Falls Sie die Schrift Flama nicht besitzen können Sie mit dieser Option auf die Schrift Arial umschalten. \tabularnewline
		\texttt{noserifmath}		& Formeln werden ebenfalls serifenlos gesetzt. \tabularnewline
		\texttt{nosectionpages} & Die Sektionseinleitungsseiten werden ausgeblendet.\tabularnewline
		\bottomrule
\end{tabularx}
\caption{Not everyone understands German}
\end{table}
\end{frame}
\subsection{Extra}
\begin{frame}[fragile]{Additional commands}
	\begin{itemize}
		\item Call package  as
		\begin{lstlisting}
		\usepackage[displaysection]{mabeamertheme}
		\end{lstlisting}
		to display section name on each frame
		\begin{alertblock}[\bigexclaim]
			Frames with \texttt{lstlisting} as in the gray box above requier 
			 \texttt{\textbackslash begin\{frame\}[fragile]}
		\end{alertblock}
		
		\item For remarks, 	use \texttt{remarkblock}s:
		\begin{equationblock}[Pythagoras]
			a^2+b^2=c^2
		\end{equationblock}
		\begin{remarkblock}
			\centering Extension to general triangles are possible, but are beyond scope of this presentation
		\end{remarkblock}
	\end{itemize}

\end{frame}
\end{document}
